% ----------------------------------------------------------------
% AMS-LaTeX Paper ************************************************
% **** -----------------------------------------------------------
\documentclass{article}
\usepackage{graphicx}
% ----------------------------------------------------------------
\vfuzz2pt % Don't report over-full v-boxes if over-edge is small
\hfuzz2pt % Don't report over-full h-boxes if over-edge is small
% THEOREMS -------------------------------------------------------

\newcommand{\clip}{ \textbf{Clip} }
% ----------------------------------------------------------------
\begin{document}

\section{Doc}
\subsection{Reorient}

To reorient two unit vectors, say \textbf{v$_1$} to \textbf{v$_2$} with two
rotations around the $\mathbf{\phi}$ and $\mathbf{\mathbf{\chi}}$ unit vectors,
\clip uses the following approach.

We we need to determine two rotation matrices $R_\phi$ and $R_\chi$, so that
$v_2 = R_\mathbf{\phi} \cdot R_\mathbf{\chi} \cdot v_1$, or equivalently
$R_\mathbf{\phi}^{-1} \cdot v_2 =  R_\mathbf{\chi} \cdot v_1$.
$R_\mathbf{\phi}^{-1} \cdot v_2$ and $R_\mathbf{\chi} \cdot v_1$ lie on a
circle on the unit sphere. These circles are the intersections of the unit
sphere with planes having $\phi$ and $\chi$ as normals and distances from the
origin of $v_2\cdot\phi$ and $v_1\cdot\chi$ respectively. These equations of
these planes are: $\mathbf{x}\cdot\mathbf{\phi}=v_2\cdot\mathbf{\phi}$ and
$\mathbf{x}\cdot\mathbf{\chi}=v_1\cdot\mathbf{\chi}$. The intersection of these
two planes is a line and the vectors on this line with a length of 1 represent
the possible intermediate vectors for the rotation: $v_i=R_\mathbf{\phi}^{-1}
\cdot \mathbf{v_2}$.

The equation of the line is
\begin{equation}
    u_1+\varepsilon\cdot u_2
    \label{lineEqn}
\end{equation}
where $u_2$ could be chosen to be perpendicular to $\mathbf{\phi}$ and
$\mathbf{\chi}$ and therefore to  be simply
$u_2=\mathbf{\phi}\times\mathbf{\chi}$ and $u_1$ could be chosen to lie in the
$\mathbf{\phi}\,\mathbf{\chi}$-plane. This gives $u_1=\lambda \cdot \phi + \mu
\cdot \chi$. The length of $u_1$ is the minimal distance of the line from the
origin and therefore determines the number of solutions. If it is larger than
one, the intersection of the two planes lie entirely outside of the unit sphere
and no rotation around $\phi$ and $\chi$ from $v_1$ to $v_2$ is possible. If
the length is exactly one, then there is one solution and corresponding one
possible rotation. If the length is smaller than one, are two possible
solutions.

To determine the coefficients in the representation of $u_1$ we consider the
equations for the two planes (remember that $u_2$ is perpendicular to both
$\chi$ and $\phi$ and therefore the product vanishes). Here we get the
equations  $(\lambda \cdot \phi + \mu \cdot \chi) \cdot \phi =
v_2\cdot\mathbf{\phi}$ and $(\lambda \cdot \phi + \mu \cdot \chi) \cdot \chi =
v_1\cdot\mathbf{\chi}$. The first equation simplifies to $\lambda  = (v_2-\mu
\cdot \chi) \cdot\mathbf{\phi}$ and inserted in the second gives the solutions

\begin{equation}
    \lambda =
    \frac{v_2\cdot\phi-(\phi\cdot\chi)(v_1\cdot\chi)}{1-(\phi\cdot\chi)^2}
\end{equation}

\begin{equation}
    \mu =
    \frac{v_1\cdot\chi-(\phi\cdot\chi)(v_2\cdot\phi)}{1-(\phi\cdot\chi)^2}
\end{equation}

If we now normalize $u_2$, we could rewrite equation (\ref{lineEqn}) as
$\varepsilon=\pm\sqrt{1-|u_1|^2}$


\end{document}
% ----------------------------------------------------------------
